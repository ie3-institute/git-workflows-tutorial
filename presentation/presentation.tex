% Do compile with lualatex utilizing the Makefile

\PassOptionsToPackage{unicode}{hyperref}
\documentclass[aspectratio=1610, 9pt]{beamer}

\usepackage{polyglossia}
\setmainlanguage{english}

\newcommand{\iethree}{ie\textsuperscript{3}}

\usetheme[
  showtotalframes, % show total number of frames in the footline
]{tudo}

\title{Advanced Coding Workshop}
\subtitle{How to produce good code and test it automatically}
\author[\iethree{}-Team]{O. Pohl, C. Kittl, J. Hiry, O. Kraft, D. S. Sarma, F. Rewald}
\institute[\iethree{}]{Institute of Energy Systems, Energy Efficiency and Energy Economics}

\begin{document}

\maketitle

\begin{frame}{Disclaimer}
    This presentation is a continuation of an internal training regarding git basics and workflows.
    Though it might be suitable for many application cases, we tried to fit it to the needs we have at \iethree{}.

    The audience asked for more information about testing and continous integration (CI) of those tests.
    However, we felt that a thorough basis on how to write good code is the basis for being able to write understandable and sensible tests.
    Therefore, this presentation starts with a first insight into "clean code" and sets up on this further.
\end{frame}

\begin{frame}{Contents}
    \tableofcontents
\end{frame}

\section{Introduction}
\begin{frame}{Introduction}
    \begin{itemize}
        \item Automated tests help mainting software functionality throughout changes in code base
        \item Two types of tests are known
        \begin{itemize}
            \item Unit Tests -- Testing the single bits and pieces of code
            \item Integration Tests -- Testing the interplay with outside systems
        \end{itemize}
        \item Unit tests somehow are basis for integration testint
        \item Therefore, a clean code structure and implementation is key to proper unit testing
    \end{itemize}
\end{frame}

\section{Clean Code}
\begin{frame}
    
\end{frame}

\section{Unit Tests}

\end{document}
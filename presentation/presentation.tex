% Do compile with lualatex utilizing the Makefile

\PassOptionsToPackage{unicode}{hyperref}
\documentclass[aspectratio=1610, 9pt]{beamer}

\usepackage{polyglossia}
\setmainlanguage{english}
\usepackage[
    backend = biber,
    style = ieee
]{biblatex}
\addbibresource{bib.bib}
\usepackage{listings}
\lstset{
    basicstyle = \small\ttfamily,
    frame = leftline,
    rulecolor = \color{tudark},
    backgroundcolor = \color{tudark!15}
}

\newcommand{\iethree}{ie\textsuperscript{3}}

\usetheme[
  showtotalframes, % show total number of frames in the footline
]{tudo}

\title{Advanced Coding Workshop}
\subtitle{How to produce good code and test it automatically}
\author[\iethree{}-Team]{O. Pohl, C. Kittl, J. Hiry, O. Kraft, D. S. Sarma, F. Rewald}
\institute[\iethree{}]{Institute of Energy Systems, Energy Efficiency and Energy Economics}

\begin{document}

\maketitle

\begin{frame}{Disclaimer}
    This presentation is a continuation of an internal training regarding git basics and workflows.
    Though it might be suitable for many application cases, we tried to fit it to the needs we have at \iethree{}.

    The audience asked for more information about testing and continuous integration (CI) of those tests.
    However, we felt that a thorough basis on how to write good code is the basis for being able to write understandable and sensible tests.
    Therefore, this presentation starts with a first insight into “clean code” and sets up on this further.
\end{frame}

\begin{frame}{Contents}
    \tableofcontents
\end{frame}

\section{Introduction}
\begin{frame}{Introduction}
    \textcolor{tugreen}{\textbf{Why testing?}} Automated tests help to maintain software functionality throughout changes in code base.

    \textcolor{tugreen}{\textbf{When to test?}} As soon as possible. There are two cases, when testing comes into play:
    \begin{enumerate}
        \item When you develop new code (the easy way).
        \item When a colleague (it's often Mr. or Mrs.\,{}Past You) hands you over bad code (the hard to really hard way).
    \end{enumerate}

    To be honest, testing is often a trade-off decision.
    It is early investment, but prevents you from struggeling later.
    For sure, not every small automation script needs testing.
\end{frame}

\begin{frame}{Introduction (continued)}
    \textcolor{tugreen}{\textbf{What has Clean Coding to do with testing?}} It's because of the nature of tests. There are two types out there:
    \begin{enumerate}
        \item \textit{Unit Tests} -- Testing the single bits and pieces of code
        \item \textit{Integration Tests} -- Testing the interplay with outside systems
    \end{enumerate}

    Unit tests somehow are basis for integration testing.
    Therefore, a clean and modular code structure as well as implementation is key to proper unit testing.
\end{frame}

\section{Clean Code}
\begin{frame}{Mindset}
    \begin{columns}
        \begin{column}{0.5\textwidth}
            \setlength{\parskip}{12pt}%
            Clean code is not stoically following a rule set you have learned.
            It is more of a \textcolor{tugreen}{\emph{mindset}}.

            See yourself as a craftsman!
            You are crafting something every day -- do it with joy and be proud of it.
            And most important: You get better every day!
            Picture clean code as your own target, improving every day and gaining skills -- not only for the code itself but your way of thinking and approaching problems.

            Most of what is presented here, is inspired by the work of \href{https://en.wikipedia.org/wiki/Robert_C._Martin}{Robert C. Martin}, especially \cite{Martin.2009}.
            He considers code to be clean, if it is
            \begin{center}
                \textcolor{tugreen}{\emph{simple, comprehensible and fulfills it's purpose}}.
            \end{center}
        \end{column}
        \begin{column}{0.5\textwidth}
            \includegraphics[width = \columnwidth]{figures/stock/pottery.jpg}
            {\footnotesize\textit{Picture from Enric Sagarra at \href{https://pixabay.com/images/id-3050615/}{Pixabay}}}
        \end{column}
    \end{columns}
\end{frame}

\begin{frame}{Campsite Rule}
    \begin{columns}
        \begin{column}{0.5\textwidth}
            \setlength{\parskip}{12pt}%
            Well okay, \ldots{} There are some “rules” -- better say guidelines.

            The first one is pretty simple and inspired by the Scout society's “Campsite Rule”:

            \begin{center}
                \textcolor{tugreen}{\emph{Leave everything cleaner, than it has been before.}}
            \end{center}

            Whenever you see a design flaw, bad structure or just a wrong comment:
            Try to improve this piece of code.
        \end{column}
        \begin{column}{0.5\textwidth}
            \includegraphics[trim = {100px 0 250px 0}, clip, width = \columnwidth]{figures/stock/campfire.jpg}
            {\footnotesize\textit{Picture from LUM3N at \href{https://pixabay.com/de/photos/lagerfeuer-lager-outdoor-topf-896196/}{Pixabay}}}
        \end{column}
    \end{columns}
\end{frame}

\begin{frame}[fragile]{Naming}
    \begin{columns}
        \begin{column}{0.5\textwidth}
            \setlength{\parskip}{12pt}%
            
            Use comprehensible names for your variables, attributes, methods and classes:
            
            \begin{itemize}
                \item Names should reveal the purpose
                \item Avoid acronyms
                \item Leave out details about implementations\\E.\,{}g. \lstinline{users} instead of \lstinline{usersList}
                \item Stick to known concepts\\E.\,{}g. \lstinline{resultListener} instead of \lstinline{resultCollector}
                \item Use \href{https://en.wikipedia.org/wiki/Naming_convention_(programming)}{langauge-specific naming conventions}
            \end{itemize}
        \end{column}
        \begin{column}{0.5\textwidth}
            \setlength{\parskip}{6pt}%

            \begin{lstlisting}[language = Python]
def d(a, b):
    return a/b
            \end{lstlisting}

            \begin{lstlisting}[language = Python]
def divide(nominator, denominator):
    return nominator/denominator
            \end{lstlisting}

            \begin{lstlisting}[language = Python]
a = [...] # All users
b = [...] # All co-workers

for i in b:
    d = 0.0
    e = list(filter(lambda x: x.cust == i, a))
    for f in e:
        d = d + f.dur
    
    print(i + " -> " + d)
            \end{lstlisting}
            What does happen, here?
        \end{column}
    \end{columns}
\end{frame}

\begin{frame}[fragile]{Classes and Methods}
    \begin{columns}
        \begin{column}{0.5\textwidth}
            \setlength{\parskip}{9pt}%
            
            Classes and methods are for grouping.
            Therefore, 
            \begin{center}
                \textcolor{tugreen}{\emph{one class / method per purpose.}}
            \end{center}

            \begin{itemize}
                \item Avoid side effects
                \item Don't alter the parameters, you hand in to a function
                \item Avoid toggling method behavior by input
                \item Exceptions: Split business logic and error handling
                \item “Don't give a null, don't take a null” -- Use proper values or error handling instead.
            \end{itemize}

            Well encapsulated methods and classes help you to \textcolor{tugreen}{\emph{recycle code and avoid duplications}}.
        \end{column}
        \begin{column}{0.5\textwidth}
            \setlength{\parskip}{6pt}%
            \begin{lstlisting}[language = Python]
some_array = []
sumOf = add(1, 2)
print("Sum is " + sumOf)

def add(a, b):
    c = a + b
    some_array.append(c)
    return c
            \end{lstlisting}
            
            \begin{lstlisting}[language = Python]
def add(a, b, add_positive):
    if add_positive:
        return a + b
    else:
        return a - b
            \end{lstlisting}
        \end{column}
    \end{columns}
\end{frame}

% \begin{frame}
%     \begin{columns}
%         \begin{column}{0.5\textwidth}
%             \setlength{\parskip}{12pt}%
%         \end{column}
%         \begin{column}{0.5\textwidth}
%             \setlength{\parskip}{12pt}%
%         \end{column}
%     \end{columns}
% \end{frame}

\section{Unit Tests}

\begin{frame}{References}
    \printbibliography
\end{frame}
\end{document}